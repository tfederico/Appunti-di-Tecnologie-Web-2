
\chapter{Il sito Web}

	\section{La metafora del negozio}
		Possiamo considerare qualsiasi sito web come una casa o meglio un negozio: la gente guarda e poi decide se comprare o andarsene.
		L'\textbf{homepage} è la vetrina del negozio in cui le persone cercano informazione e questa informazione deve essere usufruibile nel modo più efficace possibile. A tal proposito sorge il problema di comunicare nei migliori dei modi l'informazione, un problema fortunatamente già affrontato dal giornalismo. Il pezzo informativo perfetto è il risultato dei 5 assi informativi principali, le così dette \emph{5 W} (\emph{6 W}): Where - Who - Why - What - When (- How).
		Che nel web si traducono rispettivamente in:
		\begin{description}
			\item[Where] A quale sito sono arrivato?
			\item[Who] Chi c'è dietro questo sito?
			\item[Why] Perché sono qui? Quale benefici mi dai?
			\item[What] Che cosa mi offri? Mostramelo.
			\item[When] Ultime novità del sito.
			\item[How] Capito questo, come arrivo a quello di mio interesse?
		\end{description}
	
	\section{Problemi e implicazioni}
		Il principale problema di un utente che visita il sito è il \textbf{TEMPO}. Bisogna sempre considerare che gli utenti hanno:
			\begin{itemize}
				\item aspettative.
				\item poco tempo, secondi contati!
			\end{itemize}
		Il sito quindi deve sapere offrire le \emph{6 W} nel pochissimo tempo che l'utente gli dedica.	L'utente medio all'arrivo sulla homepage ha circa \textbf{31 secondi} prima di cominciare ad avere sensazioni negative. 31 secondi. Solo 31 secondi per convincere l'utente. Questo porta ad una serie di implicazioni:
		
			\subsection{Quanto testo nella homepage?} Un uomo adulto di buona cultura legge dalle 200 alle 300 parole al minuto, su computer però la velocità di lettura è più bassa: 180 parole al minuto. Con più di 93 parole abbiamo già superato il limite di 31 secondi, se teniamo poi conto dell'intero layout allora sono decisamente troppe.
			\subsection{Il comportamento dell'utente è dinamico} Bisogna far sì che l'utente al nostro sito ci ritorni ma ora le W di Who, Where e Why non sono più richieste. Fortunatamente l'utente salta alcuni pezzi ma ora ha ancora meno tempo da dedicare.
			
		\subsection{Questione di tempo}	
			Di seguito i tempi medi di permanenza:
			\begin{itemize}
				\item 1ª volta: 31 secondi.
				\item 2ª volta: 25 secondi.
				\item 3ª volta: 22 secondi.
				\item 4ª volta: 19 secondi.
				\item dalla 5ª volta in poi i tempi sono stabili.
			\end{itemize}
			Dalla seconda volta in poi quello è il patrimonio dei secondi da dedicare agli assi What, When e How, corrispondono a 57 parole al massimo!
			Una home prolissa non darà mai tutti gli assi informativi nei pochi secondi a disposizione e una home poco chiara (assi informativi mancanti) darà un motivo in più per scappare all'utente.
			
		\subsection{E il resto del sito?}
			Per tutte le altre pagine non abbiamo bisogno che gli assi siano il principale obiettivo informativo. Inoltre l'utente una volta superata la homepage (vetrina) dedica più tempo. Dai 31 secondi passa a \textbf{53 secondi} che corrispondono a circa 160 parole in cui includere info più specifiche.
			Un ottimo modo per gestire il poco numero di parole è quello di attirare l'attenzione dell'utente con descrizioni corte che conducano ad altre pagine per ulteriori informazioni. Ciò fa sì che solo l'utente effettivamente interessato leggerà il testo più lungo mentre agli altri verrà fatto perdere meno tempo.
			Sembrerebbe un'ottima soluzione quella di spezzare la pagina e resettare i timer guadagnando tempo ma attenzione perché oltre al tempo singolo di ogni pagina bisogna considerare anche il \textbf{tempo globale}.
			
		\subsection{Questione di tempo - Parte II}
			Il \textbf{tempo globale} rapresenta il tempo massimo dell'utente per raggiungere lo scopo. Si suddivide in due:
			\begin{description}
				\item[Tempo preliminare:] è il tempo che un utente dedica per convincersi a restare nel sito, per questo chiamato anche \textbf{tempo di scelta}. Il tempo di scelta medio è di 1 minuto e 50 secondi, allo scadere di questo timer l'utente abbandona il sito indipendentemente se esso conteneva l'informazione ricercata o no. Nell'88\% dei casi quell'utente non ritornerà più.
				\item[tempo complessivo:] l'utente è convinto a restare per cui dedica fino a 3 minuti e 49 secondi per avere successo altrimenti abbandona.
			\end{description}
			
			\subparagraph*{Morale:}
			\begin{quote}
				``È molto importante il bilanciamento tra homepage e pagine interne"
			\end{quote}
			Al primo accesso infatti l'utente dopo aver navigato homepage e una pagina interna decide se restare o andarsene (1:50 - tempo di scelta). Dopo tre pagine e mezzo l'utente deve aver successo in quello che doveva fare.
	
	\section{L'importanza della struttura}
		Abbiamo capito:
		\begin{itemize}
			\item dopo un click l'utente deve essere convinto a restare;
			\item dopo tre link l'utente deve avere quello che cercava.
		\end{itemize}
		
		Ora nelle pagine interne che assi informativi servono? Sembrerebbe che non serva replicare le info della homepage nelle pagine interne ma la navigazione al giorno d'oggi non attraversa quasi mai la homepage!
		Grazie ai motori di ricerca infatti la navigazione comincia da qualunque punto di qualsiasi sito.
	
		\subsection{Deep linking}
			Questo fenomeno viene chiamato \textbf{deep linking} ovvero avere il link interno di un sito. Accade questo perché anche i motori di ricerca hanno i loro timer e devono dare nel modo più diretto l'informazione giusta che l'utente cerca.
		Ogni pagina quindi può essere una pagina iniziale per un utente. La metafora del negozio si fa critica perché ciò significherebbe avere clienti teletrasportati all'interno, già tra gli scaffali.
		
		\subsection{Gli assi in dettaglio}
			Andiamo quindi a vedere gli assi in dettaglio e come bisogna gestirli in seguito al \emph{deep linking}.
			Alcuni assi risultano essere \textbf{obbligatori} per tutte le pagine:
			\begin{itemize}
				\item Who: il logo (solitamente da preferire in alto a sinistra).
				\item What: tipicamente un link alla homepage.
			\end{itemize}
			Altri risultano essere \textbf{opzionali}:
			\begin{itemize}
				\item When: le novità del sito.
			\end{itemize}
			Altri ancora entrano nella categoria \textbf{opzionali consigliati}:
			\begin{itemize}
				\item Why: basta una breve descrizione, uno slogan.
				\item How: funzionalità di search (da prefreire in alto a sinistra).
			\end{itemize}
		
		\subsection{L'importanza del Where}
			Un paragrafo a parte invece è doveroso dedicarlo all'asse Where. Infatti in ogni pagina si dovrebbe rendere chiaro il contesto in cui l'utente si trova. Si potrebbe obiettare con "perché non mandarlo alla homepage" come per l'asse What ma ciò costituirebbe un link in più all'utente e, peggio, spostare l'utente dal luogo in cui c'è l'informazione di suo interesse. Conviene dargli informazioni del where nella pagina interna.
				Per fare questo si utilizza il \emph{breadcrump}, ne esistono di tre tipi:
				\begin{description}
					\item[Location:] dà il posto della pagina nella gerarchia del sito. Ad esempio: "Home >> Categoria >> Pubblicità >> Pagina".
					\item[Attribute:] mostra la categoria e gli attributi della pagina. Un po' come gli hashtag odierni. Un pagina può trovarsi sotto più categorie.
					\item[Path:] mostrano il cammino dell'utente per giunger alla pagine. È dinamico infatti dipende dal cammino dell'utente e usa dei \emph{cookie} per tenere traccia di tali informazioni.
				\end{description}
				
				\paragraph{Pro e contro}
					\begin{itemize}
						\item \textbf{Path} non risolve il problema del Where dopo che l'utente è catapultato nella pagina.
						\item \textbf{Attribute} sembra la scelta migliore ma implica un sistema più complesso per gestire il sito e raggiunge taglie troppo grandi in certi casi.
						\item \textbf{Path} resta una soluzione semplice e lineare.
					\end{itemize}
				
				\paragraph{Separatori}
					Per completezza si riportano i separatori per \emph{breadcrump} più comuni:
					\begin{itemize}
						\item segno di maggiore ">";
						\item segno di doppio maggiore ">>";
						\item backslash "\textbackslash";
					\end{itemize}
				